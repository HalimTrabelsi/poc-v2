\documentclass[10pt, letterpaper]{article}

% ===== PAGE GEOMETRY =====
\usepackage[
    ignoreheadfoot,
    top=1.0cm,
    bottom=1.0cm,
    left=1.2cm,
    right=1.2cm,
    footskip=0.8cm
]{geometry}

% ===== PACKAGES =====
\usepackage{titlesec}
\usepackage{tabularx}
\usepackage{array}
\usepackage[dvipsnames]{xcolor}
\usepackage{enumitem}
\usepackage{fontawesome5}
\usepackage{amsmath}
\usepackage[
    pdftitle={Halim Trabelsi - CV},
    pdfauthor={Halim Trabelsi},
    colorlinks=true,
    urlcolor=MidnightBlue
]{hyperref}
\usepackage[pscoord]{eso-pic}
\usepackage{calc}
\usepackage{bookmark}
\usepackage{lastpage}
\usepackage{changepage}
\usepackage{paracol}
\usepackage{ifthen}
\usepackage{needspace}
\usepackage{iftex}

% ===== FONTS =====
\ifPDFTeX
    \input{glyphtounicode}
    \pdfgentounicode=1
    \usepackage[T1]{fontenc}
    \usepackage[utf8]{inputenc}
    \usepackage{lmodern}
\fi

\usepackage{charter}

% ===== STYLE =====
\definecolor{primaryColor}{RGB}{0, 51, 102}

\raggedright
\pagestyle{empty}
\setcounter{secnumdepth}{0}
\setlength{\parindent}{0pt}
\setlength{\columnsep}{0.25cm}
\setlength{\topskip}{0pt}
\pagenumbering{gobble}

\titleformat{\section}
{\needspace{4\baselineskip}\bfseries\large\color{primaryColor}}
{}{0pt}{}[\vspace{1pt}\titlerule]
\titlespacing{\section}{-1pt}{0.12cm}{0.1cm}

\newenvironment{highlights}{
    \begin{itemize}[
        topsep=0.05cm,
        parsep=0.03cm,
        itemsep=0pt,
        leftmargin=10pt,
        labelsep=4pt
    ]
}{
    \end{itemize}
}
\renewcommand\labelitemi{\fontsize{5pt}{5pt}\selectfont$\bullet$}

\newenvironment{onecolentry}{
    \begin{adjustwidth}{0cm}{0cm}\fontsize{10pt}{11.5pt}\selectfont
}{
    \end{adjustwidth}
}

\newenvironment{twocolentry}[2][]{
    \begin{paracol}{2}
    \setlength{\columnsep}{0.25cm}
    \def\secondColumn{#2}
    \setcolumnwidth{\fill, 3.8cm}
    \fontsize{10pt}{11.5pt}\selectfont
}{
    \switchcolumn \raggedleft \secondColumn
    \end{paracol}
}

% ===== DOCUMENT =====
\begin{document}

% ===== 1. HEADER =====
\begin{center}
    \fontsize{18pt}{18pt}\selectfont \textbf{Halim Trabelsi}

    \vspace{3pt}

    \small
    \faMapMarker*\, Ariana, Tunisie
    \kern 6pt|\kern 6pt
    \faEnvelope\, \href{mailto:halim.trabelsi@esprit.tn}{halim.trabelsi@esprit.tn}
    \kern 6pt|\kern 6pt
    \faPhone*\, \href{tel:+21623340490}{+216 23 340 490}
    \kern 6pt|\kern 6pt
    \faLinkedin\, \href{https://www.linkedin.com/in/halim-trabelsi-a49a74246/}{linkedin.com/in/halim-trabelsi}
\end{center}

\vspace{5pt}

% ===== 2. PROFILE =====
\section{Profil Professionnel}
\begin{onecolentry}
Ingénieur en informatique passionné par le \textbf{développement full-stack} et l’\textbf{intégration de solutions IA}. 
Spécialisé en \textbf{Software Engineering – Technologies Web et Internet} à \textbf{ESPRIT}. 
Solide maîtrise des environnements \textbf{MERN}, \textbf{Spring Boot}, \textbf{Django} et \textbf{DevOps (Docker, CI/CD)}. 
Motivé par la conception d’applications web performantes, intelligentes et centrées sur l’expérience utilisateur.
\end{onecolentry}

\vspace{3pt}

% ===== 3. OBJECTIF =====
\section{Objectif}
\begin{onecolentry}
Recherche d’un \textbf{stage de fin d’études (PFE)} en \textbf{développement full-stack et intégration de solutions IA}, 
avec opportunité de \textbf{pré-embauche} dans un environnement technologique innovant.
\end{onecolentry}

\vspace{3pt}

% ===== 4. SKILLS =====
\section{Compétences Techniques}
\begin{onecolentry}
\textbf{Langages \& Frameworks :} Java (JEE, Spring Boot), JavaScript (React, Node.js, Angular), PHP (Laravel, Symfony), Python (Django, Flask), HTML5, CSS3 \\
\textbf{Bases de Données :} MySQL, SQLite, MongoDB, SPARQL, PostgreSQL \\
\textbf{Développement Web :} MERN Stack, RESTful APIs, Microservices, OAuth2, JWT Auth, WebSocket \\
\textbf{DevOps :} Git, GitHub/GitLab, Jenkins, Docker, SonarQube, Docker Compose, CI/CD Pipelines \\
\textbf{Intelligence Artificielle :} Python (Pandas, Scikit-learn, TensorFlow), intégration AI \& automatisation \\

\textbf{Design UI/UX :} Figma, Responsive Web Design, Wireframing, Prototypage \\
\textbf{Gestion de Projet :} Agile/Scrum, Jira, Trello, Xray, Documentation technique
\end{onecolentry}

\vspace{3pt}

% ===== 5. EXPERIENCE =====
\section{Expérience Professionnelle}

\begin{twocolentry}{Juil. 2022}
\textbf{Stage de formation humaine et sociale – EY Tunisia}
\end{twocolentry}
\begin{onecolentry}
\begin{highlights}
    \item Développement d’une application web \textbf{SmartHealthTracker} avec \textbf{Laravel}.
    \item Conception d’un module de \textbf{suivi d’activités quotidiennes} : To-Do List, calendrier interactif et catégorisation.
\end{highlights}
\textbf{Technologies :} PHP (Laravel), HTML, CSS, JS, MySQL
\end{onecolentry}

\vspace{4pt}

\begin{twocolentry}{Juin – Août 2024}
\textbf{Stage d’immersion – Akwel Automotive}
\end{twocolentry}
\begin{onecolentry}
\begin{highlights}
    \item Développement d’un \textbf{système de tickets automatisé} pour le service IT.
    \item Mise en place de \textbf{dashboards dynamiques} et d’un \textbf{chat interne en temps réel}.
    \item Sécurisation via \textbf{JWT} et \textbf{PHP Sockets}.
\end{highlights}
\textbf{Technologies :} PHP, MySQL, JWT, WebSocket, SMTP, HTML, CSS, JS
\end{onecolentry}

\vspace{4pt}

\begin{twocolentry}{Juil. – Août 2025 (prévu)}
\textbf{Stage Ingénieur – EY Tunisia : Intégration d’OpenG2P sur Odoo 17}
\end{twocolentry}
\begin{onecolentry}
\begin{highlights}
    \item Déploiement d’une \textbf{plateforme d’aides sociales} sous \textbf{Odoo 17 / OpenG2P}.
    \item Configuration d’une \textbf{infrastructure Docker multi-conteneurs}.
    \item Intégration des modules Registry, Program et Accounting.
    \item Conception d’un \textbf{chatbot intelligent (Botpress)} pour assister les utilisateurs.
\end{highlights}
\textbf{Technologies :} Odoo 17, OpenG2P, Docker, PostgreSQL, Botpress, GitHub
\end{onecolentry}

\vspace{4pt}

% ===== 6. PROJECTS =====
\section{Projets Académiques et Techniques}

\begin{twocolentry}{Jan. – Mai 2024}
\textbf{Application de gestion des hôpitaux militaires – ESPRIT}
\end{twocolentry}
\begin{onecolentry}
\begin{highlights}
    \item Développement d’un \textbf{site web et desktop app} pour la gestion hospitalière.
    \item Système intelligent de planification et gestion de stock.
    \item \textit{Projet nominé au “Bal des Projets 2024”.}
\end{highlights}
\textbf{Technologies :} Symfony, JS, Bootstrap, MySQL
\end{onecolentry}

\vspace{4pt}

\begin{twocolentry}{Sept. – Nov. 2024}
\textbf{Automatisation d’un pipeline CI/CD – ESPRIT}
\end{twocolentry}
\begin{onecolentry}
\begin{highlights}
    \item Mise en place d’un \textbf{pipeline CI/CD} complet avec \textbf{Maven, Docker, SonarQube, Kubernetes}.
\end{highlights}
\textbf{Technologies :} Spring Boot, Docker, Kubernetes, Terraform, Nexus
\end{onecolentry}

\vspace{4pt}

\begin{twocolentry}{Jan. – Mai 2025}
\textbf{Finova – Smarter Accounting for Tunisia}
\end{twocolentry}
\begin{onecolentry}
\begin{highlights}
    \item Plateforme de comptabilité intelligente avec \textbf{IA, chatbot, reconnaissance faciale et NLP}.
    \item \textit{Projet nominé au “Bal des Projets 2025”.}
\end{highlights}
\textbf{Technologies :} ReactJS, Node.js, MongoDB, Python (AI/ML), CI/CD
\end{onecolentry}

\vspace{4pt}

\begin{twocolentry}{Sept. – Oct. 2025 (prévu)}
\textbf{Plateforme santé et fitness intelligente – Full-Stack Web \& IA}
\end{twocolentry}
\begin{onecolentry}
\begin{highlights}
    \item Application \textbf{React + Node.js + Python} avec API sécurisée JWT.
    \item Moteur IA de recommandation basé sur \textbf{SPARQL}.
\end{highlights}
\textbf{Technologies :} ReactJS, Node.js, Python, SPARQL, WebSocket
\end{onecolentry}

\vspace{4pt}

\begin{twocolentry}{Jan. – Mai 2025}
\textbf{Projet IA-Learning – Plateforme web éducative intelligente (ESPRIT)}
\end{twocolentry}
\begin{onecolentry}
\begin{highlights}
    \item Plateforme web collaborative assistée par \textbf{IA} pour la révision et l’évaluation.
    \item Backend \textbf{Django/MongoDB} + \textbf{Sneat Template}.
    \item Génération automatique de QCM et quiz adaptatifs.
\end{highlights}
\textbf{Technologies :} Django, MongoDB, Sneat, AI APIs, Twilio
\end{onecolentry}

\vspace{4pt}

% ===== 7. EDUCATION =====
\section{Formation}
\begin{twocolentry}{2021}
\textbf{Baccalauréat en Mathématiques}
\end{twocolentry}
\begin{onecolentry}
Lycée 2 Mars – Mateur, Bizerte
\end{onecolentry}

\vspace{2pt}
\begin{twocolentry}{2021 – 2026}
\textbf{Diplôme d’Ingénieur en Informatique – Technologies du Web et de l’Internet}
\end{twocolentry}
\begin{onecolentry}
ESPRIT – École Supérieure Privée d'Ingénierie et de Technologies, Ariana
\end{onecolentry}

\vspace{4pt}

% ===== 8. CERTIFICATIONS =====
\section{Certifications}
\begin{onecolentry}
\begin{highlights}
    \item \textbf{AWS Academy Graduate – Cloud Foundations} (AWS, 2025)
    \item \textbf{Hashgraph Developer Attendance Certificate} – The Hashgraph Association (2025)
    \item \textbf{Scrum Master Professional Certificate (SMPC)} – CertiProf
\end{highlights}
\end{onecolentry}

\vspace{3pt}

% ===== 9. LANGUAGES =====
\section{Langues}
\begin{onecolentry}
Arabe (Langue maternelle) \,|\, Français (B2 – professionnel) \,|\, Anglais (B2 – technique)
\end{onecolentry}

\vspace{3pt}

% ===== 10. SOFT SKILLS =====
\section{Soft Skills}
\begin{onecolentry}
Esprit d’équipe, autonomie, curiosité technologique, sens du détail, communication proactive
\end{onecolentry}

\end{document}
